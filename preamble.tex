% ------------------------------------------------------------ %
% TIPO Y PAQUETES                                              %
% ------------------------------------------------------------ %

\documentclass[11pt, a4paper]{report}
\usepackage[spanish, es-tabla, activeacute]{babel}
\usepackage[pagestyles]{titlesec}
\usepackage[hidelinks]{hyperref}
\usepackage{graphicx}
\usepackage{geometry}
\usepackage{indentfirst}
\usepackage[table,xcdraw]{xcolor}
\usepackage{verbatim}
\usepackage{keyval}
\usepackage{etoolbox}
\usepackage{fontspec}
\usepackage{titling}
\usepackage{tikz}
\usepackage{caption}
\usepackage{chngcntr}
\usepackage{listings}
\usepackage{float}
\usepackage{graphicx}
\usepackage{pifont}
\usepackage{booktabs}
\usepackage[bottom]{footmisc}
\usepackage{multirow}
\usetikzlibrary{positioning}

\definecolor{eus-color}{gray}{0.4}
\definecolor{eng-color}{gray}{0.5}

\definecolor{code-background}{rgb}{0.95,0.95,0.92}
\definecolor{code-comments}{gray}{0.6}
\definecolor{code-color}{gray}{0.3}

\newcommand*{\noaddvspace}{\renewcommand*{\addvspace}[1]{}}
\addtocontents{lof}{\protect\noaddvspace}
\addtocontents{lot}{\protect\noaddvspace}

% ------------------------------------------------------------ %
% NUMERACIÓN CONTINUA DE FIGURAS Y TABLAS                      %
% ------------------------------------------------------------ %

\counterwithout{figure}{chapter}
\counterwithout{table}{chapter}

% ------------------------------------------------------------ %
% PARTICIÓN DE PALABRAS AL FINAL DE LÍNEA                      %
% ------------------------------------------------------------ %

\pretolerance=2000
\tolerance=3000

% ------------------------------------------------------------ %
% FUENTES DEL DOCUMENTO                                        %
% ------------------------------------------------------------ %

 \setmainfont[
 Path          = fonts/,
 Ligatures     = TeX,
 UprightFont   = EHUSans-Light.otf,
 BoldItalicFont= EHUSans-BoldItalic.otf,
 BoldFont      = EHUSans-Bold.otf,
 ItalicFont    = EHUSans-LightItalic.otf
 ]{EHUSans}
 
 \setmonofont[
 Path          = fonts/,
 UprightFont   = UbuntuMono-Regular.ttf,
 BoldItalicFont= UbuntuMono-BoldItalic.ttf,
 BoldFont      = UbuntuMono-Bold.ttf,
 ItalicFont    = UbuntuMono-Italic.ttf
 ]{UbuntuMono}

\newfontfamily{\sansthick}[
 Path          = fonts/,
 BoldFont      = EHUSans-Bold.otf,
 UprightFont   = EHUSans-Regular.otf
 ]{EHUSans}
 
 \newfontfamily{\sansblack}[
 Path          = fonts/,
 UprightFont   = EHUSans-Black.otf
 ]{EHUSans}
 
 \newfontfamily{\serif}[
 Path          = fonts/,
 Ligatures     = TeX,
 UprightFont   = EHUSerif-Light.otf,
 BoldItalicFont= EHUSerif-BoldItalic.otf,
 BoldFont      = EHUSerif-Bold.otf,
 ItalicFont    = EHUSerif-LightItalic.otf
 ]{EHUSerif}
  
 \newfontfamily{\serifthick}[
 Path          = fonts/,
 BoldFont      = EHUSerif-Bold.otf,
 UprightFont   = EHUSerif-Regular.otf
 ]{EHUSerif}
 
 \newfontfamily{\serifblack}[
 Path          = fonts/,
 UprightFont   = EHUSerif-Black.otf
 ]{EHUSerif}
 
% ------------------------------------------------------------ %
% BLOQUES DE CÓDIGO                                            %
% ------------------------------------------------------------ %

\lstset{
    backgroundcolor=\color{code-background},
    commentstyle=\color{code-comments},
    basicstyle=\ttfamily\small\color{code-color},
    framexrightmargin=5pt,
    framexleftmargin=5pt,
    framextopmargin=6pt,
    framexbottommargin=6pt, 
    frame=tb, framerule=0pt,
}

% ------------------------------------------------------------ %
% ETIQUETAS EN LAS TABLAS Y FIGURAS							   %
% ------------------------------------------------------------ %

\DeclareCaptionFont{ehu}{\fontsize{9}{10}\mdseries}
\usepackage[font={ehu}, labelfont={bf}]{caption}

% ------------------------------------------------------------ %
% AÑADIR CUARTO NIVEL DE TÍTULO                                %
% ------------------------------------------------------------ %

\setcounter{secnumdepth}{4}
\titleformat{\paragraph}
{\normalfont\normalsize\bfseries}{\theparagraph.}{1em}{}
\titlespacing*{\paragraph}
{0pt}{3.25ex plus 1ex minus .2ex}{1.5ex plus .2ex}

% ------------------------------------------------------------ %
% ELIMINAR CABECERA DE CAPÍTULOS                               %
% ------------------------------------------------------------ %

\titleformat{\chapter}%
  {\normalfont\bfseries\Huge}{\thechapter.}{10pt}{}
\newpagestyle{mystyle}{
  \sethead[][\thechapter.\enspace\chaptertitle][]{}{\thesection~\sectiontitle}{}
\setfoot{}{\thepage}{}}

% ------------------------------------------------------------ %
% MÁRGENES Y DISTANCIA ENTRE PÁRRAFOS						   %
% ------------------------------------------------------------ %

\geometry{top=2.5cm,left=3cm,right=3cm,bottom=2.5cm}
\setlength{\parskip}{8pt}

% ------------------------------------------------------------ %
% TÍTULO DEL ÍNDICE                                            %
% ------------------------------------------------------------ %

\addto\captionsspanish{
	\renewcommand{\contentsname}{Índice}
	\renewcommand{\listtablename}{Lista de tablas}
	\renewcommand{\listfigurename}{Lista de figuras}
}

% ------------------------------------------------------------ %
% DIRECTORIO DE IMÁGENES                                       %
% ------------------------------------------------------------ %

\graphicspath{{images/}}

% ------------------------------------------------------------ %
% FORMATO DEL TÍTULO                                           %
% El documento principal debe incluir este código:             %
% Título:
% \newcommand{\titulo}{<Título del trabajo>}
% % Nombre del Máster:
% \newcommand{\tituloMaster}{XXXXXXXXXX} 
% % Alumno/Alumna:
% \newcommand{\estudiante}{<Apellido 1, Apellido 2, Nombre>}
% % Dirección del TFG:
% \newcommand{\direccion}{<Apellido 1, Apellido 2, Nombre>}
% % Departamento:
% \newcommand{\departamento}{<Departamento>}
% % Curso académico:
% \newcommand{\curso}{<Curso>} 
% % Fecha:
% \newcommand{\fecha}{Bilbao, X de XXXX de 2020}
% ------------------------------------------------------------ %

\newcommand{\portada}{
	{\raggedleft\includegraphics[width=8cm]{corp/ehu}}
	\\
	{\LARGE\textbf{\MakeUppercase{Máster universitario en}}}\\[0.5\baselineskip]
	{\LARGE\textbf{\MakeUppercase{\tituloMaster}}}
	\\[3\baselineskip]
	\begin{center}
		\Huge{\textbf{\MakeUppercase{Trabajo fin de Máster}}}
		\\[1\baselineskip]
		\begin{tikzpicture}
			\renewcommand{\baselinestretch}{0.75}
			\node (A) at (0.03cm,-0.03cm) [inner sep=20pt, draw, minimum width=\linewidth, text width=13cm, minimum height=104pt, fill=black, align=center]
			{\huge{\MakeUppercase{\textbf{\textit{\titulo}}}}};
			\node (B) at (0,0) [inner sep=20pt, draw, minimum width=\linewidth, text width=13cm, minimum height=104pt, fill=white, align=center] 
			{\huge{\MakeUppercase{\textbf{\textit{\titulo}}}}};
		\end{tikzpicture}
	\end{center}
	\renewcommand{\baselinestretch}{1.25}
	\begin{tikzpicture}[remember picture, overlay]
		\node (C) at ([yshift=-4.95cm, xshift=14.5cm]current page.west) [inner sep=10pt, thick, minimum width=13.34cm, minimum height=64pt, align=left, fill=white]{};
		\node (D) [right=0.2cm of C.west, anchor=west, align=left]
		{\small{\textbf{Estudiante: }\estudiante}\\
			\small{\textbf{Director/a: }\direccion}\\
			\small{\textbf{Departamento: }\departamento}\\
			\small{\textbf{Curso académico: }\curso}
			};
	\end{tikzpicture}
	\renewcommand{\baselinestretch}{1}
	\begin{tikzpicture}[remember picture, overlay]
		\node (G) at ([yshift=-10.6cm, xshift=14cm]current page.west) [inner sep=10pt, thick, minimum width=7.93cm, minimum height=27pt, align=left, fill=white]{};
		\node (H) [left=0.2cm of G.east, anchor=east]
		{\small{\textit{\fecha}}};
	\end{tikzpicture}
}
